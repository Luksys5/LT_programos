\documentclass[a4paper,12pt]{article}
\usepackage[utf8x]{inputenc}
\usepackage[T1]{fontenc}

%\usepackage[T2A]{fontenc} % jei yra kirilica
\usepackage[hmargin={30mm,15mm},vmargin={20mm,20mm},bindingoffset=0mm]{geometry}
\usepackage[onehalfspacing]{setspace}
\usepackage[colorlinks=true, linkcolor=blue, citecolor=blue, urlcolor=blue, unicode]{hyperref}

%\parindent=7mm
\renewcommand{\refname}{Literatūros sąrašas} % article
%\renewcommand{\bibname}{Literatūros sąrašas} % report
\renewcommand{\contentsname}{Turinys}
\usepackage[T1]{fontenc} 

% Lukas paketai
\usepackage{booktabs}% http://ctan.org/pkg/booktabs
\newcommand{\tabitem}{~~\llap{\textbullet}~~}
\usepackage{graphicx}
\usepackage{indentfirst}
\usepackage{setspace}
\usepackage{color}
\usepackage{placeins}
\usepackage{booktabs}% http://ctan.org/pkg/booktabs
\usepackage{tabularx}% http://ctan.org/pkg/tabularx
\usepackage[parfill]{parskip}
\usepackage[unicode]{hyperref}
\usepackage{hyperref}
\usepackage{tocloft}
\usepackage{graphicx}
\newcommand\AtPageUpperRight[1]{\AtPageUpperLeft{%
   \makebox[\paperwidth][r]{#1}}}
\usepackage[dotinlabels]{titletoc}
\usepackage[capposition=top]{floatrow}
\hypersetup{
    colorlinks,
    citecolor=black,
    filecolor=black,
    linkcolor=black,
    urlcolor=black
}
\usepackage{secdot}




\begin{document}
\graphicspath{ {/} }

\renewcommand{\cftdot}{.}	
\renewcommand{\cftsecleader}{\cftdotfill{\cftdotsep}}

\thispagestyle{empty} % nerasomas psl. nr


\begin{center}
\textbf{Darbų portfolio} \\
\end{center}
\vspace{0.5cm}
\normalsize
\section{Work in repositories}

\textbf{Python}
\begin{itemize}
	\item \href{https://github.com/Luksys5/LT_programos/tree/code}{Hamming, Shannon file coding}
	\item 
\href{https://github.com/Luksys5/LT_programos/tree/Bakalaurinis}{Bachelour programs.}
	\item 
\href{https://github.com/Luksys5/LT_programos/tree/Biotrees}{Programs from bioinformatic lectures.}
	\item 
\href{https://github.com/Luksys5/LT_programos/tree/tinklai}{Programs for Computer networks.}
	\item 
\href{https://github.com/Luksys5/LT_programos/tree/Duomenu_Tyrimas}{Data analysis.} \\\\
\end{itemize}

\textbf{C, C++, C\#}
\begin{itemize}
	\item 
\href{https://github.com/Luksys5/LT_programos/tree/Dirbtini_Iq}{Programs for AI lectures.}
	\item 
\href{https://github.com/Luksys5/LT_programos/tree/Bioinformatika_4k}{Programs from bioinformatic lectures.} \\\\
\end{itemize}

\textbf{Perl, Shell, Bash}
\begin{itemize}
	\item 
\href{https://github.com/Luksys5/LT_programos/tree/GNU-PERL}{Programs from bioinformatic lectures.}\\\\ 
\end{itemize} 


\textbf{HTML, CSS, Js, PhP, Mysql}
\begin{itemize}
	\item 	\href{https://github.com/Luksys5/LT_programos/tree/Tinklapiai}{Website project development.}\\\\ 
\end{itemize}

\clearpage

\section{Carry out work description}

\textbf{C, C++, C\#}
\begin{enumerate}
	\item Multiprogramming operation system project(C). \\
	Task: Create virtual, real machine. Project was developed from 3 individual group
	\item Algorithms realization and object movement(C++).\ Different AI programs: tower of Hannoi, backtracking, forward and backward chaining. Developed with javascript molecular movement and with C++ Lagrange polynomial - tasks for bioinformatics lecture.
	\item Game development(C\#).\\
	Developed a game using Unity engine. Game development process - interesting becouse of the it's creation and gathered knowledge about other games creations.
Project was developed from 3 individual group.

\end{enumerate}
	

\textbf{HTML, CSS, Javascript, PHP, Mysql }
\begin{enumerate}
	\item website development project - Lithuanian cities\\
	This project wasn't fully finished. Yet I gathered lot's of knowledge and intersting information about websites creation basics. 
\end{enumerate}

\textbf{Python}
\begin{enumerate}
	\item Bachelour and course work.\\
	It's goal was to gather biological structural data and then analize it. So the part of analyzation was done with Python.
	\item Other studies work.
		\begin{itemize}
			\item Computer networks programs. Connect to computer and gather data from connection with other computer or website.
			\item Shannon and hamming algorithm realization in file coding. 
			\item Bioinformatics program. Two protein code seeks local alignment with Smith-Waterman algorithm.
		\end{itemize}

\end{enumerate}

\textbf{Perl, SHELL, BASH }

\begin{enumerate}
	\item Programs for calculating protein structural features.
	\item Bioinformatics and crystallographics programs and their testing with makefiles. Test files developed with shell.
\end{enumerate}


\end{document}