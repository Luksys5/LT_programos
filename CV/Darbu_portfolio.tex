\documentclass[a4paper,12pt]{article}
\usepackage[utf8x]{inputenc}
\usepackage[T1]{fontenc}

%\usepackage[T2A]{fontenc} % jei yra kirilica
\usepackage[hmargin={30mm,15mm},vmargin={20mm,20mm},bindingoffset=0mm]{geometry}
\usepackage[onehalfspacing]{setspace}
\usepackage[colorlinks=true, linkcolor=blue, citecolor=blue, urlcolor=blue, unicode]{hyperref}

%\parindent=7mm
\renewcommand{\refname}{Literatūros sąrašas} % article
%\renewcommand{\bibname}{Literatūros sąrašas} % report
\renewcommand{\contentsname}{Turinys}
\usepackage[T1]{fontenc} 

% Lukas paketai
\usepackage{booktabs}% http://ctan.org/pkg/booktabs
\newcommand{\tabitem}{~~\llap{\textbullet}~~}
\usepackage{graphicx}
\usepackage{indentfirst}
\usepackage{setspace}
\usepackage{color}
\usepackage{placeins}
\usepackage{booktabs}% http://ctan.org/pkg/booktabs
\usepackage{tabularx}% http://ctan.org/pkg/tabularx
\usepackage[parfill]{parskip}
\usepackage[unicode]{hyperref}
\usepackage{hyperref}
\usepackage{tocloft}
\usepackage{graphicx}
\newcommand\AtPageUpperRight[1]{\AtPageUpperLeft{%
   \makebox[\paperwidth][r]{#1}}}
\usepackage[dotinlabels]{titletoc}
\usepackage[capposition=top]{floatrow}
\hypersetup{
    colorlinks,
    citecolor=black,
    filecolor=black,
    linkcolor=black,
    urlcolor=black
}
\usepackage{secdot}




\begin{document}
\graphicspath{ {/} }

\renewcommand{\cftdot}{.}	
\renewcommand{\cftsecleader}{\cftdotfill{\cftdotsep}}

\thispagestyle{empty} % nerasomas psl. nr


\begin{center}
\textbf{Darbų portfolio} \\
\end{center}
\vspace{0.5cm}
\normalsize
\section{Darbai, esantys repozitorijoje}

\textbf{Python}
\begin{itemize}
	\item \href{https://github.com/Luksys5/LT_programos/tree/code}{Hamming, Shannon failų kodavimas}
	\item 
\href{https://github.com/Luksys5/LT_programos/tree/Bakalaurinis}{Kursinio algoritmas}
	\item 
\href{https://github.com/Luksys5/LT_programos/tree/Biotrees}{Bioinformatikos paskaitų programos}
	\item 
\href{https://github.com/Luksys5/LT_programos/tree/tinklai}{Kompiuterinių tinklų programos}
	\item 
\href{https://github.com/Luksys5/LT_programos/tree/Duomenu_Tyrimas}{Duomenų tyrimas.} \\\\
\end{itemize}

\textbf{C, C++, C\#}
\begin{itemize}
	\item 
\href{https://github.com/Luksys5/LT_programos/tree/Dirbtini_Iq}{Dirbtinio intelekto paskaitų programos.}
	\item 
\href{https://github.com/Luksys5/LT_programos/tree/Bioinformatika_4k}{Bioinformatikos paskaitų programos.} \\\\
\end{itemize}

\textbf{Perl, Shell, Bash}
\begin{itemize}
	\item 
\href{https://github.com/Luksys5/LT_programos/tree/GNU-PERL}{Bioinformatikos paskaitų programos.}\\\\ 
\end{itemize} 


\textbf{HTML, CSS, Js, PhP, Mysql}
\begin{itemize}
	\item 	\href{https://github.com/Luksys5/LT_programos/tree/Tinklapiai}{Internetinių svetainių projektas.}\\\\ 
\end{itemize}

\clearpage

\section{ATLIKTŲ DARBŲ APRAŠYMAS}

\textbf{C, C++, C\#}
\begin{enumerate}
	\item Multiprograminės operacinės sistemos projektas (C). \\
	Užduotis: Sukurti virtualią, realią mašiną su reikiamais komponentais ir sujungti šiuos darbus. Projekto realizavimas atliktas dirbant 3 asmenų grupėje. 
	
	\item Algoritmų realizavimas bei objektų judėjimas (C++). 
	Skirtingos A.I. programos: Hannojaus bokšai, išėjimo iš labirinto kūrimas.
Suprogramuotas ir tinklalapyje atvaizduotas molekulių judėjimas – užduotis bioinformatikos paskaitai.
Suprogramuotas bei R programa atvaizduotas Lagranžo polynomas.
	\item Žaidimų kūrimas (C\#).\\
	Naudojantis Unity programa sukurtas žaidimas. 
Žaidimo kūrimo procesas - įdomus, ne vien dėl rezultato – veikiančio prototipo, bet ir dėl mokymosi bei kūrimo procese atsirandančio suvokimo, kaip sukuriami populiariausi pasaulio žaidimai.  
Projektas buvo atliekamas 3 asmenų grupėje, todėl jo metu įgyta darbo komandoje įgūdžių.
	\item Mikrovaldiklių valdymo kūrimas (C\#).\\
	Naudojantis Visual Studio buvo programuojamos atskiros dalys, kurių funkcionalumas galiausiai susisteminamas į vieną galutinį produktą. \\\\
Tiek mikrovaldiklių valdymas, tiek žaidimų kūrimas – įdomus, iššūkių pilnas procesas. Jame įdomausia tai, kad rezultatas-realus, matoma, kaip veikia sukurta dalis.\\\\

\end{enumerate}


\textbf{HTML, CSS, Javascript, PHP, Mysql }
\begin{enumerate}
	\item Internetinių technologijų projektas – Lietuvos miestai.\\
Lietuvos miestų projektas - nebaigtas. Jis nepateisino didelių lūkesčių, tačiau jo metu buvo išmokta praktiškai pritaikyti svetainių kūrimo pagrindai.
	\item Svetainė - molekulinių savybių skaičiavimams.\\
	Šį projektą kurti buvo daug lengviau bei įdomiau. Problemos, su kuriomis buvo susidurta ir ankstesnio projekto metu, buvo sprendžiamos produktyviau.  \\\\
\end{enumerate}

\textbf{Python}
\begin{enumerate}
	\item Kursinis darbas.\\
	Kursinio darbo tikslas – surinkti biologinius struktūrų duomenis, juos išanalizuoti ir apskaičiuoti gautos informacijos esminius skirtumus. Iš rezultatų sudaryti bendras baltymų
	sąvybes parodyti giminingumą.
	\item Kitos studijų programos.
		\begin{itemize}
			\item Prisijungimas prie kito kompiuterio ar duomenų gavimas iš atitinkamos svetainės.
			\item Failų užkodavimas naudojant Shannon‘o ir Hamming‘o algoritmus. 
			\item Bioinformatikos programa realizuoja Smith-Waterman algoritmą – dviejų sekų.
lokalų išlyginimą. Rezultatas - sekų panašumas bei histograma.
		\end{itemize}

\end{enumerate}

\textbf{Perl, SHELL, BASH }

\begin{enumerate}
	\item Programos - skirtos apskaičiuoti biologinių struktūrų savybes.
	\item Bioinformatikos programos, testavimas su „makefile“
	\item su shell(sh) sukurti testiniai failai.
\end{enumerate}
Išvada: tik su tinkamu testavimu ištaisomos lemtingos klaidos.\\



\textbf{Patirtis}
Atlikti projektai padėjo įgyti daug naujų įgūdžių ir patobulinti turimus. Nors buvo susidurta su problemomis, tačiau jas sprendžiant išmokta nebedaryti panašių klaidų ateityje. Toliau numatoma tobulinti įgūdžius ir naudoti gautą informaciją tolimesnėje kūryboje.

\end{document}