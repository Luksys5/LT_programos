\documentclass[a4paper,12pt]{article}
\usepackage[utf8x]{inputenc}
\usepackage[T1]{fontenc}

%\usepackage[T2A]{fontenc} % jei yra kirilica
\usepackage[hmargin={30mm,15mm},vmargin={20mm,20mm},bindingoffset=0mm]{geometry}
\usepackage[onehalfspacing]{setspace}
\usepackage[colorlinks=true, linkcolor=blue, citecolor=blue, urlcolor=blue, unicode]{hyperref}

%\parindent=7mm
\renewcommand{\refname}{Literatūros sąrašas} % article
%\renewcommand{\bibname}{Literatūros sąrašas} % report
\renewcommand{\contentsname}{Turinys}
\usepackage[T1]{fontenc} 

% Lukas paketai
\usepackage{booktabs}% http://ctan.org/pkg/booktabs
\newcommand{\tabitem}{~~\llap{\textbullet}~~}
\usepackage{graphicx}
\usepackage{indentfirst}
\usepackage{setspace}
\usepackage{color}
\usepackage{placeins}
\usepackage{booktabs}% http://ctan.org/pkg/booktabs
\usepackage{tabularx}% http://ctan.org/pkg/tabularx
\usepackage[parfill]{parskip}
\usepackage[unicode]{hyperref}
\usepackage{hyperref}
\usepackage{tocloft}
\usepackage{graphicx}
\newcommand\AtPageUpperRight[1]{\AtPageUpperLeft{%
   \makebox[\paperwidth][r]{#1}}}
\usepackage[dotinlabels]{titletoc}
\usepackage[capposition=top]{floatrow}
\hypersetup{
    colorlinks,
    citecolor=black,
    filecolor=black,
    linkcolor=black,
    urlcolor=black
}
\usepackage{secdot}




\begin{document}
\graphicspath{ {/} }

\renewcommand{\cftdot}{.}	
\renewcommand{\cftsecleader}{\cftdotfill{\cftdotsep}}

\thispagestyle{empty} % nerasomas psl. nr


\begin{center}
 Curriculum Vitae\\
\textbf{Lukas Tutkus} \\

\vspace{5cm}

\textbf{Gyvenamieji adresai}
\begin{itemize}
	\item Didlaukio g. 59, Vilnius, 08302, Lietuva. 
	\item Anykščių vns., Anykščiai, 29159, Lietuva.
\end{itemize}

\textbf{Kontaktai}
\begin{itemize}
	\item \textbf{Telefonas:} 869363208
	\item \textbf{Paštas:} lt.tutkus7@gmail.com
\end{itemize}

\textbf{Bendra informacija}
\begin{itemize}
	\item \textbf{Gimimo data}: 1992 - 10 - 19
	\item \textbf{Tautybė}: Lietuvis
\end{itemize}
\vfill

Vilnius \ \  2015
\end{center}



\clearpage

\section{Asmeninės savybės}
\begin{itemize}
	\item Dirbdamas komandoje dažnai padedu rasti bendrą kalbą, nutarti galutinį variantą bei išspresti iškilusias problemas
	\item Sugebu prisitaikyti įvairiose situacijose, taip pat esu užsispyręs, tai man padeda įveikti ne vieną dieną darytą darbą.
	\item Esu diplomatiškas – priimu kritika. Stengiuosi patobulėti iš pateiktų pastabų, daryti išvadas.
	\item Nesu greitos orientacijos, todėl būčiau naudingesnis, jei iškart pasakytų ką reikia atlikti. Mėgstu tikslias aiškias užduotis
	\item Įvairiose situacijoje visada ieškau naujovių bei stengiuosi neužsibūti vienoje vietoje, judėti pirmyn.
\end{itemize}

\section{Išsilavinimas}
\textbf{2000-2012} \\
Anykščių Jono Biliūno gimnazija. Sustiprinti dalykai: biologija, matematika, fizika, informatika. \\

\textbf{2012 - dabar}
Vilniaus Matematikos ir informatikos fakulteto bioinformatikos

\textbf{Kursinis} \\
DNR replikacijos procesyvumo veiksnių ir homologų struktūrų analizė. \\
Vadovas: Darius Kazlauskas\\
Atlikta: VU Biotechnologijos instituto, Bioinformatikos skyriuje, Vilnius. 

\textbf{Bakalauro studijų metu įgytos žinios}
\begin{itemize}
	\item DNR replikacijos procesyvumo veiksnių ir homologų struktūrų analizė. Vadovas: Darius Kazlauskas, Vilniaus Universiteto Biotechnologijos institutas, Bioinformatikos skyrius, Vilnius. 
	\item Programų sistemų projektavimas, tinklapių kūrimas;
	\item Programavimo kalbų: C++, C\#, Python panaudojimas bei testavimas;
	\item Molekulių judėjimo bei sąveikos tarp jų nagrinėjimas, diferencialinių lyčių sprendimas, 
statistiniai tyrimai.
\end{itemize}

\clearpage

\vspace{8cm}


\section{Darbo įgudžiai}
\textbf{2015/01/01 iki 2015/05/31} \\
2, 3 kurse atlikinėjau praktiką Vilniaus Universiteto Biotechnologijos institute, bioinformatikos skyriuje, įgyjau pagrindines bioinformatikos žinias, bei Python programavimo pagrindus.   \\
Darbas Vilniaus Universiteto Biotechnologijos institute, bioinformatikos skyriuje. Atliktas kursinis bei vykdomas kursinio projektas, baltymų pirminės antrinės strutūros  analizė iš įvairių organizmų. 

\vspace{2cm}

\section{Pagrindiniai įgudžiai}
\textbf{Kalbos}
\begin{enumerate}
	\item Anglų kalba - C1 lygis kalbamosios kalbos, B2 rašomosios. 
	\item Vokiečių kalba - B1 lygis.
\end{enumerate}

\textbf{IT įgudžiai}
\begin{enumerate}
	\item C, C\#, C++, Java Progamavimo kalbų pagrindai (2 metai).
	\item Python, Perl programavimo kalbos (praktikos metu, 3 metai).
	\item HTML, PHP, JS (3 metai).
\end{enumerate}

\textbf{Kiti įgudžiai}
\begin{enumerate}
	\item Vim Linux aplinkoje bei Notepad++ Windows aplinkoje naudojimosi įgūdžiai.
	\item Praktika dirbant su Unity, R, Mapple, Latex.
	\item Praktika su kūrybine programa Magix Music Maker.
\end{enumerate}

\clearpage

\section{Pomėgiai}
\textbf{Sportas}
\begin{itemize}
	\item Krepšinis
	\item Dziudo
	\item Bėgimas
	\item Tenisas
\end{itemize}

\textbf{Laisvalaikio užsiėmimai}
\begin{itemize}
	\item Kompiuteriniai žaidimai (FPS, RPG, MOBA)
	\item Detektyvinės knygos
	\item stalo žaidimai (pvz.: šachmatai)
\end{itemize}

\textbf{Muzika}
\begin{itemize}
	\item Mėgstu groti pianinu, gitara.
	\item Kuri muzika su elektroniniu pianinu.
\end{itemize}

\end{document}
